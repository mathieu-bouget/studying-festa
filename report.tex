	\documentclass[11pt]{article}

	\usepackage[utf8]{inputenc}
	\usepackage[T1]{fontenc}
	\usepackage{textcomp}
	\usepackage{amsmath, amssymb}
	\usepackage[english]{babel}
	\usepackage{tikz}
	\usepackage{tikz-cd}
	\usepackage{nameref}
	\newtheorem{theorem}{Theorem}
	\tikzcdset{every label/.append style = {font = \large}}
	%\tikzcdset{every vertice/.append style = {font = \huge}}

	\author{Mathieu Bouget}
	\title{Studying FESTA}

	\begin{document}
		\maketitle

		\section{Preliminaries}
		\section{SIDH}
			The Supersingular Isogeny Diffie-Hellmann is a scheme derived from the usual
			Diffie-Hellmann exchange where we have a group $G= \langle g\rangle$ and Alice choose a
			secret integer $a$ and send $g^a$, Bob choose a secret $b$ and send $g^b$ to Alice,
			at the end they are both able to compute $g^{ab}$ which is their shared secret.
			There is a starting curve $E_0$, Alice chooses an isogeny $\varphi_A:E_0\to E_A$
			and send $E_A$ to Bob who sent in the same time the curve $E_B$ he obtained
			from the isogeny $\varphi_B:E_0\to E_B$ he had chosen before.
			This protocol requires additionnal information because it's not as simple as the
			case in group where the knowledge of $g^b$ and $a$ make Alice able to construct
			$g^{ab}=(g^b)^a$.
			Alice knows $\varphi_A$ and  $E_B$ and would want to create something like
			$\varphi_A':E_B\to E_{AB}$

			%TODO sidh description
			\subsection{High-level}
			Before going into more details about the mathematical tools (2-dimension isogenies)
			and strategies to break a generic SIDH instance, I introduce a generic condition
			on the degree of the secret isogeny and the torsion of $E_0$ that is sufficient
			to find $\varphi$.
			\begin{theorem}{\cite{sidhpolyattack}}
				Let $\varphi:E_0 \to E$ be a secret degree disogeny (where $d$
				is known) and assume we are given the images of $\varphi$ on a basis $\{P,Q\}$ of $E_0[N]$,
				where $N$ and $d$ are assumed smooth and coprime, and $N^2 > 4d$. Let $\mathbb{F}_q$ be the
				smallest field over which $E_0[N],E_0[d]$ and $\varphi$ are defined, then the kernel of
				$\varphi$ can be computed in a polynomial number of operations in $\mathbb{F}_q$.
			\end{theorem}
			%TODO verify (see paper)
			In fact this attack require only $N^2>d$ to run but may return an ambiguous result
			in this case. Intuitively, the larger $N$ the more information is given about
			the behaviour of $\varphi$ and the larger $d$ is the more complex $\varphi$ is and
			so is difficult to recover from torsion points.
			$\varphi$.
			\subsection{Mathematical tools}
			Giving the images of torsion points was considered safe until a mathematical result based
			on elliptic curve product and higher-dimensionnal isogenies gave ways to recover
			the secret isogeny in classical polynomial-time.
			We recall here a result that make torsion-point attacks possibles.
			%TODO trouver formulation du thm
			\begin{theorem}
				\label{thm:twoiso}
				Let $E_0$, $E_1$ and $2$ be three elliptic curves definied over
				$\overline{\mathbb{F}}_p$ such that there exist two isogenies
				$\varphi_{N_1}:E_0\to E_1$ and $\varphi_{N_2}:E_0\to E_2$ of coprime degrees
				$\deg(\varphi_{N_1})=N_1$ and $\deg(\varphi_{N_2})=N_2$. Then, the subgroup
				$$\{([N_2]\varphi_{N_1}\times[N_1]\varphi_{N_2})\circ(P,P)~|~ P\in E_0[N_1+N_2]\}$$
				is the kernel of a $(N_1+N_2,N_1+N_2)$-polarised isogeny $\Phi$
				having product codomain $E_0\times F$ and matrix form
				$$
				\begin{pmatrix}\widehat\varphi_{N_1} & -\widehat\varphi_{N_2}
				\\ g_{N_2} & \widehat{g}_{N_1} \end{pmatrix}$$
				where $g_{N_i}$ are $N_i$-isogenies such that the following diagram commutes

% https://tikzcd.yichuanshen.de/#N4Igdg9gJgpgziAXAbVABwnAlgFyxMJZABgBpiBdUkANwEMAbAVxiRAFEB9YkAX1PSZc+QigCM5KrUYs2XAEx8BIDNjwEiEsVPrNWiEADElgtSKJlt1XbINcxfKTCgBzeEVAAzAE4QAtkhkIDgQSBLSemwAOlH03mgAFlicwABynPK8JiA+-oHUIUgAzNYy+iAxcYnJaZxiWdQMdABGMAwACkLqoiDeWC4JONm5AYglwaGI4TbllXTxScO+o-IFk9NlbC4p6fUgjS1tnWYaBn0DQ-xey8VrSKsRtiDbtZn7IE2tHV3mZ-2Djl4QA
				\begin{center}\begin{tikzcd}[sep = 5em]
E_0 \arrow[r, "\varphi_{N_2}"] \arrow[d, "\varphi_{N_1}"'] & E_2                     \\
E_1 \arrow[ru, "\varphi"] \arrow[r, "g_{N_2}"']            & F \arrow[u, "g_{N_1}"']
\end{tikzcd}\end{center}
			\end{theorem}
			
			
			In \cite{directkeysidh} and in attacks on SIDH, discussion are about finding an
			isogeny $\varphi_f$ in order to use this theorem, once the isogeny has been found
			we juste need to evaluate a two-dimensions isogeny in order to recover the kernel
			of the secret isogeny $\varphi_A$.

			
		\section{FESTA}
		In 2023, a public key encryption scheme based on the SIDH attack was introduced
		in \cite{FESTA} and relies on a trapdoor function.
		\subsection{High-level description}
		The public parameters are a supersingular elliptic curve $E_0$ definied over
		$\mathbb{F}_{p^2}$ and a basis $\mathcal B_{E_0[2^b]}=(P_b,Q_b)$ of the $2^b$-torsion
		of $E_0$.

		The key generation consists in the choice of a secret isogeny $\varphi_A$ from 
		$E_0$ to a curve $E_A$, as in SIDH some additionnal information is given about
		$\varphi_A$. Since the knowledge of $\varphi_A|_{E[2^b]}$ (ie
		$(\varphi_A(P_b),\varphi_A(Q_b))$ lets an attacker able to recover $\varphi_A$,
		the information about behaviour of  $\varphi_A$ on the  $2^b$-torsion is scaled
		with a matrix $\mathbf A$ which belon to the trapdoor with $\varphi_A$.
		Thus, the trapdoor is $(\varphi_A,\mathbf A)$ and the public key is 
		$\mathtt{pk}=(E_A,R_A,S_A)$ where
	\[
		\begin{pmatrix} R_A \\ S_A \end{pmatrix} = \mathbf A \cdot \varphi_A
	\begin{pmatrix} P_b \\ Q_b \end{pmatrix} 	
	\] 	

	% https://tikzcd.yichuanshen.de/#N4Igdg9gJgpgziAXAbVABwnAlgFyxMJZABgBpiBdUkANwEMAbAVxiRAFEB9YkAX1PSZc+QigCM5KrUYs2XAIJ8BIDNjwEiZMVPrNWiDpzFLBakUQnbqu2Qa4AmPlJhQA5vCKgAZgCcIAWyQyEBwIJAlpPTYAHWj6HzQACyxORX5vP0DECNCkAGZrGX0QWPiklMd0kF8AoOpcxHtCqINSugTkoydeIA
	\begin{center}
	\begin{tikzcd}[sep = 5em]
E_0 \arrow[r, "\varphi_A"] \arrow[d, "\varphi_1"] & E_A \arrow[d, "\varphi_2"] \\
E_1                                               & E_2
\end{tikzcd}
\end{center}

		\paragraph{Evaluation} The inputs of the evaluation function are two isogeny
		$\varphi_1:E_0\to E_1$ and $\varphi_2:E_A\to E_2$ plus a scaling matrix $\mathbf B$.
		Using a representation of the isogenies using a generator of their kernel described
		in a deterministicaly generated basis, the domain of $f_\mathtt{pk}$ is
		$$\mathbb{Z}/d_1\mathbb{Z}\times\mathbb{Z}/d_2\mathbb{Z}\times\mathcal M_b$$
		where $\mathcal M_b$ is the set of scaling matrices.
		The output of the function are curves $E_1$ and $E_2$ plus additionnal information
		about $\varphi_1$ and $\varphi_2$, that is some torsion points scaled with $\mathbf B$.
		The output is $(E_1,R_1,S_1,E_2,R_2,S_2)$ where
		\[
		\begin{pmatrix} R_1 \\ S_1 \end{pmatrix} = \mathbf B \cdot \varphi_1 
		\begin{pmatrix} P_b \\ Q_b \end{pmatrix}\quad\quad
		\begin{pmatrix} R_2 \\ S_2 \end{pmatrix} = \mathbf B \cdot\varphi_2
		\begin{pmatrix} R_A \\ S_A \end{pmatrix} = \mathbf B\cdot\varphi_2\cdot\mathbf A
		\cdot\varphi_A \begin{pmatrix} P_b \\ Q_b \end{pmatrix} 
		\] 
		\paragraph{Inversion} We have
		\begin{eqnarray*}
			(\varphi_2\circ\varphi_A\circ\widehat\varphi_1)\begin{pmatrix}R_1\\S_1\end{pmatrix} 
		&=&\varphi_2\circ\varphi_A\circ\widehat\varphi_1\cdot\mathbf B\varphi_1
		\begin{pmatrix} P_b \\ Q_b \end{pmatrix}  \\
		&=& \mathbf B \cdot\varphi_2\circ\varphi_A\circ[\widehat\varphi_1\circ\varphi_1]
		\begin{pmatrix} P_b \\ Q_b \end{pmatrix} \\
		&=&[d_1] \mathbf B\cdot\varphi_2\circ\varphi_A\begin{pmatrix} P_b \\ Q_b \end{pmatrix} \\
		&=&[d_1] \mathbf{BA^{-1}}\cdot\varphi_2\circ\mathbf A\varphi_A
		\begin{pmatrix} P_b \\ Q_b \end{pmatrix} \\
		&=&[d_1] \mathbf{BA^{-1}}\cdot\varphi_2
		\begin{pmatrix} R_A \\ S_A \end{pmatrix} \\
		&=&[d_1] \mathbf{BA^{-1}}\cdot\mathbf{B}
		\begin{pmatrix} R_2 \\ S_2 \end{pmatrix} \\
		&=&[d_1] \mathbf{A^{-1}}\cdot\begin{pmatrix} R_2 \\ S_2 \end{pmatrix} \\
	\end{eqnarray*}
	Then we can recover the isogeny $\psi=\varphi_2\circ\varphi_A\circ\widehat\varphi_1$
	using the SIDH attack, extraction of isogenies  $\varphi_1$ and $\varphi_2$ follows
	(and $\mathbf B$) using the knowledge of $\varphi_A$.

	\subsection{Low-level description}

	A pure SIDH on $\psi$ would be too expensive to be used in a decryption algorithm so
	there is a specific construction of $\varphi_A$ during the setup phase in order to make
	the attack more efficient. More precisely, there will not be any research phase of a
	suitable isogeny to construct the 2-dimension isogeny as it is the case in the general
	SIDH attack. During the setup, two isogenies are actually generated such that their
	degrees fill the equality $m_1^2d_{A,1}d_1+m_2^2d_{A,2}d_2=2^b$ for some $m_1,m_2,b\in\mathbb{Z}$
	and from these we compute $\varphi_A$.
	 \[
		 \varphi_A = \varphi_{A,2}\circ\varphi_{A,1}
	\] 
	Note that no information about $\varphi_{A,1}$, $\varphi_{A,2}$ (such at $\varphi_{A,1}(P_b)$)
	is public so it doesn't change the security of the protocol.
	To recover $\varphi_1$ and $\varphi_2$ we can use the Theorem \ref{thm:twoiso} with
	$\varphi_{N_1}=[m_1]\circ\varphi_1\circ\widehat\varphi_{A,1}$ and
	$\varphi_{N_2}=[m_2]\circ\varphi_2\circ\varphi_{A,2}$.

	\begin{center}
		% https://tikzcd.yichuanshen.de/#N4Igdg9gJgpgziAXAbVABwnAlgFyxMJZABgBpiBdUkANwEMAbAVxiRAB128HZgBRAL4B9AIIgBpdJlz5CKAIzkqtRizZ8hAJnGSQGbHgJEy85fWatEIDfJ1SDsootPVzaqwDFxymFADm8ESgAGYAThAAtkhkIDgQSIoqFmyc9KFoABZYQsAAcloCdiBhkdHUcUiarqqWHOxpmdl5QvKF1Ax0AEYwDAAK0oZyIKFYfhk4RSVRiFWx8YgAzNXJVn45+ZqFEiHh00tzCcvuIGvNrSDtXT39DkZWI2MTAhQCQA
		\begin{tikzcd}[sep = 5em]
\tilde{E}_A \arrow[r, "\varphi_{N_2}"] \arrow[d, "\varphi_{N_1}"'] & E_2                     \\
E_1 \arrow[r, "g_{N_2}"']                                           & F \arrow[u, "g_{N_1}"']
\end{tikzcd}
	\end{center}

	We have $N_1 + N_2 = m_1^2d_1d_{A,1} + m_2^2d_2d_{A,2}=2^b$ so $\tilde{E}_A[N_1+N_2]$ is
	the $2^b$-torsion of $\tilde E_A$. We have, with $P'_b=\varphi_{A,1}(P_b)$ and 
	$Q'_b=\varphi_{A,1}(Q_b)$.
	\begin{eqnarray*}
		[N_2]\varphi_{N_1}(P'_b) & = & [N_2][m_1]\circ\varphi_1\circ\tilde\varphi_{A,1}(P'_b) \\
														 & = & [N_2][m_1]\circ\varphi_1\circ\tilde\varphi_{A,1}
														 \circ\varphi_{A,1}(P_b) \\
														 & = & [N_2][m_1][d_{A,1}]\circ\varphi_1(P_b) \\
														 & = & [m_2^2d_{A,2}d_2][m_1][d_{A,1}]\circ\varphi_1(P_b) \\
														 & = & [m_2m_1d_{A,1}][m_2d_{A,2}d_2]\circ\varphi_1(P_b) \\
		N_1\varphi_{N_2}(P'_b) & = & [N_1][m_2]\circ\varphi_2\circ\varphi_{A,2}(P'_b) \\
													 & = & [N_1][m_2]\circ\varphi_2\circ\varphi_{A,2}\circ\varphi_{A,1}(P_b) \\
													 & = & [N_1][m_2]\circ\varphi_2\circ\varphi_A(P_b) \\
													 & = & [N_1][m_2]\circ\varphi_2\circ\varphi_A(P_b) \\
														& = & [m_1^2d_{A,1}d_1][m_2]\circ\varphi_2\circ\varphi_A(P_b) \\
														& = & [m_2m_1d_{A,1}][m_1d_1]\circ\varphi_2\circ\varphi_A(P_b) 
	\end{eqnarray*}
	The know the $\varphi_2\circ\varphi_A(P_b)$ and $\varphi_1(P_b)$ using $\mathbf A^{-1}$
	up to the matrix $\mathbf B$ but
	it's sufficient to have the whole kernel of $\phi$ the 2-isogeny which has matrix form :
	$$\begin{pmatrix}
		 \varphi_{N_1} & \widehat\varphi_{N_2} \\
		 \_ & \_ \end{pmatrix}
	$$

	%connaissance des degrés suffisantes ?
	

	\section{Attacks}
	The security of FESTA is based on the hardness of isogeny interpolation when the
	images of the torsion are scaled with some matrix. Breaking this problem would breaks
	the entire protocol, however there is a stronger assumption that is made in FESTA which
	is that there are two interpolation with the same scaling matrix.

	\subsection{Computational isogeny with scaled-torsion problem}
	Even if FESTA relies on a stronges assumption that this problem, there is a major intrest
	in sutdying current strategies used to break a CIST instance. I briefly introduce
	the one provided in \cite{polyCIST}.
	
	We want to recover a secret isogeny $\varphi:E_0\to E$ of known degree $d$. We are given
	the points $S,T\in E[N]$ such that
	\[
		\begin{pmatrix} S \\ T \end{pmatrix} = \mathbf A\cdot \varphi \begin{pmatrix} P \\ Q \end{pmatrix}
	\] 
	where $(P,Q)$ is a basis of  $E_0[N]$ the $N$-torsion of $E_0$ and  $\mathbf A$ a secret
	matrix.

	
	A direct "SIDH" attack on $\varphi$ isn't possible here because of the scaling matrix
	$\mathbf A$. So an idea is to attack another isogeny that doesn't contain $\mathbf A$
	but still discloses information about $\varphi$.

	For instance, if $\omega$ is an endomorphism of $E_0$,
	then recovering isogeny $\psi = \varphi \circ \omega \circ \widehat\varphi$
	could give clues on $\varphi$. Intuitively, the presence of $\varphi$ and
	$\widehat\varphi$ corresponds to an application of $\mathbf A$ and then $\mathbf A^{-1}$
	so it should cancel the scaling as long as $\omega$ "commute" with this matrix.
	More formally, we require that the endomorphism $\omega$ acting on $E_0[B]$ as a matrix
	 $\mathbf M$ in the basis $(P,Q)$ that commutes with $\mathbf A$.
\[
	\omega \begin{pmatrix} P \\ Q\end{pmatrix} = \mathbf M \begin{pmatrix} P \\ Q \end{pmatrix}
\] 
This way we have, denoting $w$ the degree of $\omega$ :
	\begin{eqnarray*}
		\varphi\circ\omega\circ\widehat\varphi \begin{pmatrix} S \\ T \end{pmatrix}
		&=& \varphi\circ\omega\circ [d]  \mathbf A \begin{pmatrix} P \\ Q \end{pmatrix} \\
		&=& [d]\mathbf A \circ \varphi\circ\omega \begin{pmatrix} P \\ Q \end{pmatrix} \\
		&=& [d]\mathbf{AM} \circ \varphi \begin{pmatrix} P \\ Q \end{pmatrix} \\
		&=& [d]\mathbf{AMA^{-1}}  \begin{pmatrix} S \\ T \end{pmatrix} \\
		&=& [d]\mathbf{M}  \begin{pmatrix} S \\ T \end{pmatrix} \\
	\end{eqnarray*}
	This equality implies that it is possible to get the images of $\varphi\circ\omega
	\circ\widehat\varphi$ on the basis $(P,Q)$ of the $N$-torsion from the only knowledge
	of the matrix $\mathbf M$ and the points $(S,T)$. Then, if the degree of this isogeny
	is small enough (in fact $N^2>wd^2$), it is possible to recover it completly using SIDH attack. 
	However, recovering this isogeny
	does not always give meaningfull information about $\varphi$. Let's take the trivial
	example of $\omega=\mathrm{Id}_{E_0}$, the isogeny $\varphi\circ\omega\circ\widehat\varphi$
	is in this case $[d]$ which doesn't contain any additionnal information on $\varphi$.

	\begin{center}
		% https://tikzcd.yichuanshen.de/#N4Igdg9gJgpgziAXAbVABwnAlgFyxMJZABgBpiBdUkANwEMAbAVxiRAFEB9YkAX1PSZc+QijIBGKrUYs27AOTc+UmFADm8IqABmAJwgBbJGRA4IScdXrNWiEAB179XWgAWWENTjvtOC9QY6ACMYBgAFITwCNl0sNVc-fh19I0QTM2MAiAg0InEADjJtRjgYKUCQ8MiRGLiEz2kbNkdDGDU6PgEQPUN-U3M0rx8-REtG2TtHAHcsWFc6HEdnNw8A4NCI7CjREFj4xIpeIA
		\begin{tikzcd}[sep=5em]
E_0 \arrow[d, "\varphi"', shift right] \arrow["\omega"', loop, distance=2em, in=125, out=55] \\
E'_0 \arrow[u, "\widehat\varphi"', shift right]
\end{tikzcd}
\end{center}

We need to controll the degree of $\omega$ in order to apply the SIDH attack. There is an
improvement of this technique that reduces the degree in some cases. We now set $\omega$ to
be an isogeny from $E_0$ to some other curve $E$ and take another isogeny $\sigma_0:E_0\to E$.
We controll the endomorphism $\omega'=\widehat\sigma_0\circ\omega$ by the matrix $\mathbf M$ :
\[
	(\widehat\sigma_0\circ\omega) \begin{pmatrix} P \\ Q\end{pmatrix} = \mathbf M \begin{pmatrix} P \\ Q \end{pmatrix}
\] 
The diagram is now
\begin{center}
% https://tikzcd.yichuanshen.de/#N4Igdg9gJgpgziAXAbVABwnAlgFyxMJZARgBoAGAXVJADcBDAGwFcYkQBRAfXJAF9S6TLnyEUZYtTpNW7DgHIe-QSAzY8BIuQpSGLNok78pMKAHN4RUADMAThAC2SbSBwQkZafvYAdHw1s0AAssEBo4EOscDxpGegAjGEYABWENMRBbLDMg6IEbeydETzdncMjo4po9WUM-AHcsWCD6HD8A4NDYhKTU9VF2LJy8lTtHMtd3RAAmapkDED9HGDN6MJBEsCgkAGZyfJAxopdSmZpN7cQAWj257zqfbDMHeiVuxJS0gcMh3OM+IA
	\begin{tikzcd}[sep=5em]
E & E_0 \arrow[d, "\varphi"', shift right] \arrow[l, "\omega", bend left] \arrow[l, "\sigma_0"', bend right] \\
  & E'_0 \arrow[u, "\widehat\varphi"', shift right]
\end{tikzcd}
\end{center}

The required inequality for using SIDH attack on $\psi$ is $N^2>swd^2$ where $s$ is the degree
of $\sigma_0$. For the moment it corresponds to the previous one since  $w'=ws$, but the point
is that the knowledge of the pushforward of $\sigma_0$ through $\varphi$ eliminates $s$.
Let assume we know $\sigma$ the pushforward of $\sigma_0$ through $\varphi$, that is the isogeny
that has kernel $\varphi(\mathrm{Ker}\sigma_0)$ going from $E'_0$ to another curve $E'$
and that effectively computable in some situation.
To delete the $s$ of the conditon we need to eliminate $\sigma_0$ in $\psi = \varphi\circ\widehat
\sigma_0\circ\omega\circ\widehat\varphi$ so we let $\psi=\varphi'\circ\omega\circ\widehat\varphi$
with $\varphi'$ the pushforward of $\varphi$ through $\sigma$ (we don't need to actually compute
it, as $\varphi$ we have knowledge on it up to the scaling matrix $\mathbf A$).
We have the following diagram :
\begin{center}
	% https://tikzcd.yichuanshen.de/#N4Igdg9gJgpgziAXAbVABwnAlgFyxMJZARgBoAGAXVJADcBDAGwFcYkQBRAfXJAF9S6TLnyEUZYtTpNW7DgHIe-QSAzY8BIuQpSGLNok7Kh60VtKSae2YYX8pMKAHN4RUADMAThAC2SbSA4EEhk0vrsADoRDJ5oABZYIDRwCe44ITSM9ABGMIwACsIaYiCeWE5x6QIe3n6IoUH+yanp9VYyBiBRAO5YsHH0OFEx8YmZOXmFppqGZRVVKl6+TYHBiABM7eGGUb4wTvRJILlgUEgAzOTVIEt1AY0bNCdniAC0l1s2XRHYTj70SnGuQKRTMs3KlWMN1qSE2qwun06w3osQS8ihtwy8MQ50RkR+5X+9j4QA
	\begin{tikzcd}[sep=5em]
E \arrow[d, "\varphi'"] & E_0 \arrow[d, "\varphi"', shift right] \arrow[l, "\omega", bend left] \arrow[l, "\sigma_0"', bend right] \\
E'                      & E'_0 \arrow[u, "\widehat\varphi"', shift right] \arrow[l, "\sigma"]
\end{tikzcd}
\end{center}
The degree of $\psi$ is clearly $wd^2$ since $\varphi$ and $\varphi'$ have the same degree $d$.
Moreover we have $\varphi'\circ\sigma_0=\sigma\circ\varphi$ and so $[s]\varphi'=\sigma\circ\varphi
\circ\widehat\omega_0$.
Here is the proof that we can find an SIDH instance on $\psi$, the found equality is
the Lemma 3 of \cite{polyCIST}.
\begin{eqnarray*}
	[s]\psi\begin{pmatrix} S \\ T \end{pmatrix} & = & [s]\varphi'\circ\omega\circ
	\widehat\varphi\begin{pmatrix} S \\ T \end{pmatrix} \\
																							&=& [s]\varphi'\circ\omega\circ\widehat
																							\varphi\mathbf A\varphi
																							\begin{pmatrix} P \\ Q \end{pmatrix} \\
																							& =&[sd]\mathbf A\varphi'\circ\omega\circ
																							\begin{pmatrix} P \\ Q \end{pmatrix}  \\
																							&=&[d]\mathbf A\sigma\circ\varphi\circ\widehat\sigma_0
																							\circ\omega\begin{pmatrix} P \\ Q \end{pmatrix}  \\
																							&=&[d]\mathbf{AM}\sigma\circ\varphi
																							\begin{pmatrix} P \\ Q \end{pmatrix} \\
																							&=&[d]\mathbf{AMA^{-1}}\sigma\begin{pmatrix} S \\ T \end{pmatrix} \\
																							&=&[d]\mathbf{M}\sigma\begin{pmatrix} S \\ T \end{pmatrix} 
\end{eqnarray*}

%	It is possible to generalize this idea to the case where $\omega$ is an isogeny from $E_0$ 
%	to another curve $E'_0$, the definition of $\psi$ must be change to
%	$\varphi'\circ\omega\circ\widehat\varphi$ with $\varphi'$ linked to $\varphi$ and
%	starting from $E_0'$ and going to another curve $E'$. Since $\omega$ isn't an endomorphism
%	anymore we need to introduce another isogeny $\sigma_0$ of the same type in order to keep
%	a matrix form 

		



		\bibliographystyle{unsrt}
		\bibliography{biblio}
	\end{document}
